\documentclass[t,dvipsnames,hyperref={colorlinks,citecolor=NavyBlue,linkcolor=NavyBlue,anchorcolor=NavyBlue,urlcolor=NavyBlue}]{beamer}

\setbeamertemplate{navigation symbols}{}
\useinnertheme{circles}
% \usecolortheme{crane}

% \usepackage[T1]{fontenc}
\usepackage[utf8]{inputenc}
% \usepackage[polish]{babel}
\usepackage[british]{babel}

\usepackage[adobe-utopia]{mathdesign}
\usepackage[series=z]{libgreek}
\usepackage{tgpagella}
\usepackage[textwidth=4cm]{todonotes}
\usepackage{minted}
\usemintedstyle{colorful}
% \usemintedstyle{autumn}
% \usemintedstyle{vs}
\usepackage{tikz}
\usetikzlibrary{matrix,arrows,calc}
\tikzset{every scope/.style={>=angle 60,thick}}

\author{Marcin Szamotulski}
\institute{\insertlogo{\includegraphics[height=1cm]{iohk-logo.png}}}
\title{Monoidal Synchronisation}

\setbeamertemplate{frametitle}[default][center]

\begin{document}
\begin{frame}
    \titlepage
\end{frame}

\part{Introduction}
\frame{
  \partpage
  \tableofcontents[part=1]
}

%%%%%%%%%%%%%%%%%%%%
\section{Semigroups}
%%%%%%%%%%%%%%%%%%%%

\begin{frame}[fragile]
  \frametitle{Semigroups}
  \begin{minted}[stripall,bgcolor=NavyBlue!5!white,fontsize=\small]{haskell}
class Semigroup a where
  -- associativity: (a <> b) <> c == a <> (b <> c)
  (<>) :: a -> a -> a
  \end{minted}

Examples
  \begin{minted}[stripall,bgcolor=NavyBlue!5!white,fontsize=\small]{haskell}
-- free semigroup with one generator
instance Semigroup (NonEmpty a) where
  (a :| as) <> (b :| bs) = a :| as ++ b : bs

instance Semigroup [a] where
  (<>) = (++)

instance Semigroup a => Semigroup (Maybe a) where
  Nothing <> a = a
  a <> Nothing = a
  Just a  <> Just b = Just (a <> b)
  \end{minted}
\end{frame}

\begin{frame}[fragile]
  \frametitle{Foldable1 type class}
  \begin{minted}[stripall,bgcolor=NavyBlue!5!white,fontsize=\small]{haskell}
-- from "semigroupoids" package
class Foldable t => Foldable1 t where
  foldMap1 :: Semigroup m
           => (a -> m) -> t a -> m
  fold1 :: Semigroup m => t m -> m
  fold1 = foldMap1 id
  toNonEmpty :: t a -> NonEmpty a
  toNonEmpty = foldMap1 (:| [])
  \end{minted}

Examples
\small{The cannonical example instance is \texttt{NonEmpty}; all the instances
  require that the container is non-empty\footnote{This is a consequence of a \texttt{Semigroup} constraint rather than the \texttt{toNonEmpty} class member.}.}
  \begin{minted}[stripall,bgcolor=NavyBlue!5!white,fontsize=\small]{haskell}
instance Foldable1 NonEmpty where
  -- proof that @NonEmpty@ is a free semigroup:
  foldMap1 :: Semigroup m => (a -> m) -> NonEmpty a -> m
  foldMap1 f (a :| (b : bs)) = f a <> foldMap1 f (b :| bs)
  foldMap1 f (a :| [])       = f a
  \end{minted}
\end{frame}

%%%%%%%%%%%%%%%%%
\section{Monoids}
%%%%%%%%%%%%%%%%%

\begin{frame}[fragile]
  \frametitle{Monoids}
  \begin{minted}[stripall,bgcolor=NavyBlue!5!white,fontsize=\small]{haskell}
class Semigroup a => Monoid a where
  -- two sided identity
  -- > a <> mempty == a == mempty <> a
  mempty :: a
  \end{minted}

  Examples
  \begin{minted}[stripall,bgcolor=NavyBlue!5!white,fontsize=\small]{haskell}
instance Monoid [a] where
  mempty = []
instance Semigroup a => Monoid (Maybe a) where
  mempty = Nothing
  \end{minted}
\end{frame}

\begin{frame}[fragile]
  \frametitle{Foldable type class}
  \begin{minted}[stripall,bgcolor=NavyBlue!5!white,fontsize=\small]{haskell}
class Foldable t where
  foldMap :: Monoid m => (a -> m) -> t a -> m
  fold :: Monoid m => t m -> m
  fold = foldMap id
  toList :: t a -> [a]
  toList = foldMap (\a -> [a]) 
  \end{minted}
\end{frame}

%%%%%%%%%%%%%%%%%%%%%%
\section{FreeAlgebras}
%%%%%%%%%%%%%%%%%%%%%%

\begin{frame}[fragile]
  \frametitle{Free algebras}
  {\small
    The \texttt{FreeAlgebra} type class can capture the heart of \texttt{Foldable} and
    \texttt{Foldable1} classes and many other free structures as well.
  }
  \begin{minted}[stripall,bgcolor=NavyBlue!5!white,fontsize=\small]{haskell}
class FreeAlgebra (m :: Type -> Type) where
  returnFree  :: a -> m a
  foldMapFree :: forall d a.
                 ( AlgebraType  m d
                 , AlgebraType0 m a )
              => (a -> d) -> m a -> d
  \end{minted}
  {\small
  For more details see
  \begin{itemize}
    \item \href{https://skillsmatter.com/skillscasts/13007-lightning-talk-rethinking-freeness-through-universal-algebra}{Haskell eXchange lightning talk};
    \item \href{https://hackage.haskell.org/package/free-algebras}{free-algebras}
      package on Hackage.
  \end{itemize}
  }
\end{frame}

\section{Near Semi-Rings}
\begin{frame}
  \frametitle{Near Semi-Ring}
  \begin{definition}
    \((S, +, \cdot, 0)\) is a near semi-ring if:
    \begin{itemize}
      \item \((S, +, 0)\) is a monoid (not necessary abelian);
      \item \((S, \cdot)\) is a semigroup;
      \item \((a + b) \cdot c = a \cdot c + b \cdot c\);
      \item \(0 \cdot a = 0\).
    \end{itemize}
  \end{definition}
  Example: if \(M\) is a monoid then all monoid homorphism form a near semi-ring with:
  \begin{itemize}
    \item \textit{multiplication}: function composition
    \item \textit{addition}: pointwise addition
    \item \textit{zero}: constant function to the identity of the monoid
  \end{itemize}
\end{frame}

%%%%%%%%%%%%%%%%%%%%%%%%%%%%%%%%%%%%%%%%%%%%
\section{First-to-Finish and Last-to-Finish}
%%%%%%%%%%%%%%%%%%%%%%%%%%%%%%%%%%%%%%%%%%%%

\begin{frame}
  \frametitle{First-to-Finish / Last-to-Finish Near Semi-Ring}
\end{frame}

\begin{frame}
  \frametitle{Monoidal Last-to-Finish}
\end{frame}

\part{Case study}
\frame{
  \partpage
  \tableofcontents[part=2]
}

%%%%%%%%%%%%%%%%%%%%%%%%
\section{Design Pattern}
%%%%%%%%%%%%%%%%%%%%%%%%

\begin{frame}
  \frametitle{Design Pattern}
\end{frame}

%%%%%%%%%%%%%%%%%%%%%
\section{Multiplexer}
%%%%%%%%%%%%%%%%%%%%%

\begin{frame}
  \frametitle{Multiplexer}
\end{frame}

%%%%%%%%%%%%%%%%%%%%%%%%%%%%%%%%%%%
\section{Inbound Protocol Governor}
%%%%%%%%%%%%%%%%%%%%%%%%%%%%%%%%%%%

\begin{frame}
  \frametitle{Inbound Protocol Governor}
\end{frame}

%%%%%%%%%%%%%%%%%%%%%%%%%%%%%%%%%%%%
\section{Outbound Protocol Governor}
%%%%%%%%%%%%%%%%%%%%%%%%%%%%%%%%%%%%

\begin{frame}
  \frametitle{Outbound Protocol Governor}
\end{frame}


\begin{frame}
  \frametitle{Acknowledgement}
\end{frame}
\end{document}
